\documentclass{article}
\usepackage{amsmath,amsthm,amssymb,mathrsfs}
\usepackage[utf8]{inputenc}
\usepackage{natbib} 
\usepackage[colorlinks,linkcolor=magenta,citecolor=blue, pagebackref=true]{hyperref}
\newcommand{\theHalgorithm}{\arabic{algorithm}}
\usepackage{algorithm}
\usepackage{algorithmic}
\usepackage{graphicx}
\usepackage{booktabs}
\usepackage{setspace}
\usepackage{authblk}


\newcommand{\R}{\mathbb{R}}
\newcommand{\E}{\mathbb{E}}
\newcommand{\I}{\mathbb{I}}
\newcommand{\mle}{\textrm{mle}}
\DeclareMathOperator{\argmax}{argmax}

\newtheorem{theorem}{Theorem}
\newtheorem{lemma}{Lemma}

\newcommand{\com}[1]{\marginpar{{\begin{minipage}{0.33\textwidth}{\setstretch{1.1} \begin{flushleft} \footnotesize \color{red}{#1} \end{flushleft} }\end{minipage}}}}

\newcommand{\cvec}[2]
{
\left[\begin{array}{c}
#1 \\ #2 
\end{array}\right] 
}

\newcommand{\symmat}[3]
{
\left[\begin{array}{cc}
#1 & #2 \\ #2 & #3
\end{array}\right] 
}

\title{Off-policy Confidence Sequences}
\author[1]{Nikos Karampatziakis}
\author[1]{Paul Mineiro}
\author[2]{Aaditya Ramdas}
\affil[1]{Microsoft}
\affil[2]{Carnegie Mellon University}
\date{\today}

\begin{document}

\maketitle


\section{Leftovers}
\subsection{Fan Bound in High Dimensions}
Let $\lambda$ be a vector such that $\lambda_i \geq 0$ and
$\sum_i \lambda_i < 1$. Let $\xi$ be a random vector with
$\xi_i \geq -1$. Then
\[
\ln(1+\lambda^\top \xi) \geq \lambda^\top \xi + \frac{\ln(1-1^\top \lambda) +1^\top \lambda}{(1^\top \lambda)^2} (\lambda^\top \xi)^2
\]
\begin{proof}
(Sketch): First show that the function $g:\R\to\R$
\[
g(x) = -\frac{(1+x)\ln(1+x)}{x}
\]
is convex for $x\geq -1$, which can be checked from its second derivative.
The first derivative of this function is the function
whose monotonicity is used to prove Fan's bound. Thus
our proof continues as follows. Given $\lambda$ as above,
define  
\[
f(x) = -\frac{(1+\lambda^\top x)\ln(1+\lambda^\top x)}{\lambda^\top x}
\]
Checking that $f$ is convex reduces to checking that $g$ is convex.
The convexity of $f$ implies that $\nabla f$ is a monotone operator so
\[
(\nabla f(X) - \nabla f(u))^\top (x-u) \geq 0
\]
holds. The result follows by applying the above inequality for 
$u=-[1,1,\ldots,1]$, using that $\nabla f(x)=\frac{\ln(1+\lambda^\top x)-\lambda^\top x}{(\lambda^\top x)^2}\lambda$, and $\lambda^\top (x+1)\geq 0$
\end{proof}


\subsection{Random variables in $[0,1]$ with common mean}
\label{app:nogrid01}
The GROW strategy is cool but requires a different bet per $m$.
The Hedged strategy is not much more worse but the bets are 
independent of $m$. Therefore we expect to be able to eliminate
the search over $m$ by finding a tight relaxation of the 
confidence sequence that is analytically easy.

We start with a simple inequality valid for $\xi\geq-1$ and $0\leq \lambda\leq 1/2$
\[
\ln(1+\lambda \xi) \geq \lambda \xi+\left(\ln\left(1-\lambda\right)+\lambda\right)\cdot \xi^{2}
\]
With this the long capital process can be lower bounded by
\[
K^+(m) \geq \exp\left(\sum \lambda_i (x_i - m) + \sum_i 
\left(\ln\left(1-\lambda_i\right)+\lambda_i\right)\cdot (x_i - m)^{2}
\right)
\]
and similarly the short capital process 
\[
K^-(m) \geq \exp\left(\sum \lambda_i (m-x_i) + \sum_i 
\left(\ln\left(1-\lambda_i\right)+\lambda_i\right)\cdot (x_i - m)^{2}
\right)
\]
Therefore, for the Hedged process we have
\[
K^\pm(m) \geq 
\exp\left(\sum_i 
\left(\ln\left(1-\lambda_i\right)+\lambda_i\right)\cdot (x_i - m)^{2}
\right)
\cosh\left(\sum \lambda_i (x_i - m)\right)
\]
We also take advantage of $\ln\cosh(x)\geq |x| -\ln(2)$
to write
\[
\ln K^\pm(m)\geq 
\sum_i  (\ln\left(1-\lambda_i\right)+\lambda_i) \cdot(x_i-m)^2 + \left|\sum\lambda_i (x_i - m)\right| - \ln(2)
\]
If this bound is greater that $\ln(1/\alpha)$ for some $m$ then that $m$ is clearly
outside the confidence sequence for the hedged capital process. Therefore it remains
to find $m$ such that
\[
\sum_i  (\ln\left(1-\lambda_i\right)+\lambda_i) \cdot(x_i-m)^2 + \left|\sum\lambda_i (x_i - m)\right| = \ln(2/\alpha)
\]
Let
\begin{align*}
C&=\sum_i \lambda_i x_i\\
S&=\sum_i \lambda_i\\
Q&=\sum_i \left(\ln(1-\lambda_i)+\lambda_i\right) x_i^2\\
T&=\sum_i \left(\ln(1-\lambda_i)+\lambda_i\right) x_i\\
U&=\sum_i \left(\ln(1-\lambda_i)+\lambda_i\right)\\
\end{align*}
The candidate $m$ solve
\[
\left|C - Sm \right| + Q-2Tm+Um^2 = \ln(2/\alpha)
\]
and in particular the lower bound satisfies $C-Sm \geq 0$ and solves
\[
Um^2 - (2T+S) m + Q+C-\ln(2/\alpha)=0
\]
and the upper bound satisfies $C-Sm \leq 0$ and solves
\[
Um^2 - (2T-S)m + Q-C-\ln(2/\alpha)=0
\]
If an equation has no roots we take $m=0$ for the 
lower bound or $m=1$ for the upper bound.
When we have two roots we use the 
tightest (for the lower bound
we take the largest, for the upper bound the smallest). 
Taking into account that $U<0$ this means
\[
m_{\min} = \frac{2T+S-\sqrt{(2T+S)^2-4U(Q+C-\ln(2/\alpha))}}{2U}
\]
\[
m_{\max} = \frac{2T-S+\sqrt{(2T-S)^2-4U(Q-C-\ln(2/\alpha))}}{2U}
\]
For extremely large experiments and production grade code, 
it is advised to use the alternative formulas
\[
m_{\min} = \frac{2(Q+C-\ln(2/\alpha))}{2T+S+\sqrt{(2T+S)^2-4U(Q+C-\ln(2/\alpha))}}
\]
\[
m_{\max} = \frac{2(Q-C-\ln(2/\alpha))}{2T-S-\sqrt{(2T-S)^2-4U(Q-C-\ln(2/\alpha))}}
\]
which are numerically stable when $2T+S$ or $2T-S$ has very large magnitude. 

\end{document}